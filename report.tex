% Evaluation of Programmatically-Accessible Hosting Services for Visual Aids
% for an Online Learning Environment
% Copyright (C) 2012  Michael Chang
% based on
% uw-wkrpt-se.tex - An example work report that uses uw-wkrpt.cls
% Copyright (C) 2002,2003  Simon Law
% 
% This program is free software; you can redistribute it and/or modify
% it under the terms of the GNU General Public License as published by
% the Free Software Foundation; either version 2 of the License, or
% (at your option) any later version.
% 
% This program is distributed in the hope that it will be useful,
% but WITHOUT ANY WARRANTY; without even the implied warranty of
% MERCHANTABILITY or FITNESS FOR A PARTICULAR PURPOSE.  See the
% GNU General Public License for more details.
% 
% You should have received a copy of the GNU General Public License
% along with this program; if not, write to the Free Software
% Foundation, Inc., 59 Temple Place, Suite 330, Boston, MA  02111-1307  USA
%
%%%%%%%%%%%%%%%%%%%%%%%%%%%%%%%%%%%%%%%%%%%%%%%%%%%%%%%%%%%%%%%%%%%%%
%
% We begin by calling the workreport class which includes all the
% definitions for the macros we will use.
\documentclass[se]{uw-wkrpt}

% LaTeX preamble: load some packages to add functionality
\usepackage{graphicx} % Include graphic importing

\usepackage[T1]{fontenc} % Better fonts
\usepackage{ae,aecompl}

\usepackage{indentfirst} % Indent first paragraph of each section

\usepackage[titletoc,title]{appendix} % Prefix appendix letters with `Appendix'

% For mathematical symbols in our pseudocode
\usepackage{amsmath}

% Use the algorithmicx package for pseudocode
\usepackage{algorithm}
\usepackage{algpseudocode}

% Use biblatex for references
\usepackage[style=ieee,sorting=none,dateabbrev=false,backend=biber]{biblatex}
\addbibresource{report.bib} % Specify the bibliography file

\usepackage{listings}

% This needs to be the last package loaded
\usepackage[pdftex]{hyperref} % Generate PDF links and bookmarks.
\hypersetup{
  bookmarks=true,
  bookmarksnumbered=true
}

% Now we will begin writing the document.
\begin{document}

%%%%%%%%%%%%%%%%%%%%%%%%%%%%%%%%%%%%%%%%%%%%%%%%%%%%%%%%%%%%%%%%%%%%%
%% IMPORTANT INFORMATION
%%%%%%%%%%%%%%%%%%%%%%%%%%%%%%%%%%%%%%%%%%%%%%%%%%%%%%%%%%%%%%%%%%%%%

%% First we, should create a title page.  This is done below:
% Fill in the title of your report.
\title{ Evaluation of Programmatically-Accessible Hosting Services for Visual 
Aids for an Online Learning Environment }

% Fill in your name.
\author{Michael Chang}

% Fill in your student ID number.
\uwid{20332377}

% Fill in the name of the PNG file with your signature, or leave unchanged.
\signature{signature}

% Fill in your home address.
\address{30 Doncrest Rd.\\*
         Richmond Hill, ON\ \ L4B 1A2}

% Fill in your employer's name.
\employer{Khan Academy}

% Fill in your employer's city and province.
\employeraddress{Mountain View, CA}

% Fill in your school's name.
\school{University of Waterloo}

% Fill in your faculty name.
\faculty{Software Engineering}

% Fill in your student user ID
\userid{m9chang}

% Fill in your e-mail address.
\email{m9chang@uwaterloo.ca}

% Fill in your term.
\term{3B}

% Fill in your program.
\program{Software Engineering}

% Fill in the department chair's name.
\chair{Dr.\ A.\ Morton}

% Fill in the department chair's mailing address.
\chairaddress{Software Engineering\\*
              University of Waterloo\\*
	      Waterloo, ON\ \ N2L 3G1}

% If you are writing an "SE-confidential" report, uncomment the next line.
%\confidential{SE-confidential}

% If you want to specify the date, fill it in here.  If you comment out
% this line, today's date will be substituted.
\date{May 11, 2012}

% Now, we ask LaTeX to generate the title.
\maketitle

%%%%%%%%%%%%%%%%%%%%%%%%%%%%%%%%%%%%%%%%%%%%%%%%%%%%%%%%%%%%%%%%%%%%%
%% FRONT MATTER
%%%%%%%%%%%%%%%%%%%%%%%%%%%%%%%%%%%%%%%%%%%%%%%%%%%%%%%%%%%%%%%%%%%%%
%% \frontmatter will make the \section commands ignore their numbering,
%% it will also use roman page numbers.
\frontmatter

% After this, we must create a letter of submission.
\begin{letter}
I have completed my fifth work term, following my \theterm{} term. Please find
enclosed my fourth work term report entitled: ``\thetitle'' for \theemployer, a 
non-profit organization with the mission of providing "a free world-class 
education for everyone everywhere". My manager was B. Eater, and our team was 
primarily responsible with creating interactive content ("Exercises") on the 
site.

This report focuses on selecting a service for hosting images to be used in 
exercises on Khan Academy. This problem is one that I encountered on my work 
term.

I wish to thank B. Eater and B. Alpert for their advice on the technical 
content of this report. I also wish to thank C. Kleynhans and M. Yee for 
recommending the use of the uw-wkrpt \LaTeX class to format this report.

% Note that I do not need to type out the boilerplate confirmation,
% nor do I need to write a signature block.  This is generated for me.
% We are now finished with the letter.
\end{letter}

% We continue with required sections, such as the Executive Summary.
\section{Executive Summary}
Khan Academy provides exercises to help learners practice and review concepts 
that they've learned. Historically, these exercises have been built using the 
BSD-licensed khan-exercises framework. It is difficult to find and retain 
people to contribute exercises using the framework, as they both need to know 
how to teach and how to program in JavaScript.

Khan Academy is developing new tools to allow teachers to contribute to 
exercises without knowing how to program in JavaScript. These tools allow 
teachers to build a large bank of questions, which can be bundled together into 
an exercise. The initial implementation of the question editor only allowed 
users to embed images by hotlinking them from somewhere else.

This document describes the selection of a hosting service for the images that 
are uploaded in the question editor. It describes the criteria by which the 
services were evaluated, and uses the multi-criteria decision-making 
methodology to rank the services.

% Next, we need to make a Table of Contents, List of Figures and 
% List of Tables.  You will most likely need to run LaTeX twice to
% get these correct.  The first pass for LaTeX to figure out the
% labels, and the second pass to put in the right references.
\tableofcontents
\listoffigures
\listoftables

%%%%%%%%%%%%%%%%%%%%%%%%%%%%%%%%%%%%%%%%%%%%%%%%%%%%%%%%%%%%%%%%%%%%%
%% REPORT BODY
%%%%%%%%%%%%%%%%%%%%%%%%%%%%%%%%%%%%%%%%%%%%%%%%%%%%%%%%%%%%%%%%%%%%%
%% \main will make the \section commands numbered again,
%% it will also use arabic page numbers.
\mainmatter

% You must have an Introduction
\section{Introduction}\label{sec:intro}
Khan Academy provides exercises to help learners practice and review concepts 
that they've learned. Historically, these exercises have been built using the 
BSD-licensed khan-exercises framework, with each problem being randomly 
generated. This works well for exercises that focus on computation, such as 
exercises for basic addition, vectors, and matrices. However, it is cumbersome 
for word problems in math, and even more so for other subjects, such as 
biology, chemistry, math, or history. Furthermore, people who wish to 
contribute exercises must both be proficient in teaching and in programming in 
JavaScript -- only a small fraction of contributions received meet both content 
and technical standards to be incorporated into Khan Academy.

\subsection{Static Questions}\label{sec:staticq}
Khan Academy is developing new tools to allow teachers to contribute to 
exercises without knowing how to program in JavaScript. These tools allow 
teachers to build a large bank of questions, and bundle questions together into 
exercises. These tools will enable more teachers to contribute content to Khan 
Academy, which will help Khan Academy reach complete coverage of High School 
mathematics sooner. These tools will also enable content creators to create 
exercises in topics, such as World War I, that are unsuitable for the 
programmatic creation of question content. This report focuses on one 
particular component of this toolkit -- the question editor.

The question editor uses a combination of Markdown syntax for formatting, and 
LaTeX syntax for mathematical formulas. The initial version of the question 
editor allowed images to be inserted by using the syntax shown in 
figure~\ref{fig:imgsyntax}. However, the syntax provided only allowed users to 
embed images by hotlinking them from another server, such as Dropbox.

\begin{figure}
  \centering
  \lstinputlisting{imgsyntax.md}
  \caption{Static Questions - Image Embed Syntax.}
  \label{fig:imgsyntax}
\end {figure}

\subsection{Problem Statement}\label{sec:problem}
Hotlinking is considered poor practice for a variety of reasons. In particular, 
it reduces site performance for Khan Academy users, due to the need for 
additional DNS lookups and HTTP connections. It also puts undue financial costs 
on the owner of the other server. In order to remove the need for content 
creators to hotlink images, Khan Academy needs to procure a hosting service for 
the images used in the static questions.

This report seeks to select an appropriate hosting provider for images included 
in static questions on Khan Academy.

% You must have either an Analysis or a Synthesis section.
\section{Analysis}
First, we develop criteria and consider various hosting providers. We use the 
weighted sum model,\footnote{This method is covered in MSCI 261, a core course 
in the software engineering curriculum at the University of Waterloo. If you 
need a refresher, you can read about it on Wikipedia at
\url{http://en.wikipedia.org/wiki/Weighted_sum_model}.} a method of
multi-criteria decision making, to select an alternative.

\subsection {Criteria}
Criteria were selected based on past experience and overall organization 
priorities.

\subsubsection{Existing Relationship With Provider}
Do we already pay for services from this provider? Ranked from 1 (no prior 
relationship) to 5 (already use and pay for their services for mission-critical 
loads).

\subsubsection{Team Familiarity}
How much work would a developer need to learn how to use this service? Ranked 
from 1 (would need extensive training) to 5 (previously used in 
mission-critical loads).

\subsubsection{User Impact}
How do we think the user experience will be impacted by this alternative?
Questions are served thousands of times a day, and are written to / edited 
infrequently. Are the URLs for accessing resources stored on this service 
cacheable? Ranked from 1 (need to generate a new URL for every user) to 5 (URLs 
can be cached for at least one year).

\subsubsection{Tools}
How easy is it to administrate and manage the content hosted on the service?  
Ranked from 1 (requires assistance from hosting provider) to 5 (well-maintained 
web console and third-party tools available).

\subsubsection{Programmatic Access}
How easy is it to programmatically upload content to the service?
Ranked from 1 (web upload only) to 5 (documented API, with maintained libraries 
available, and revokable API keys for automated agents)

\subsubsection{Custom ACLs Support}
Ranked from 1 (binary account-wide full-access/no-access) to 5 (fine-grained 
per-file ACLs).

\subsection{Alternatives}
Possible solutions were generated by asking current employees about services 
they have used in the past (either professionally or personally).

\subsubsection{S3}
Provided by Amazon.

\subsubsection{Blob store (Google App Engine)}
Provided by Google.

\subsubsection{Storage API}
Provided by Google

\subsubsection{Imgur}
Provided by Imgur.

\subsubsection{Dropbox}
Provided by Dropbox.

\subsection{Multi-Criteria Decision-Making}
We note that the MCDM usually isn't very robust if the criteria weights are 
changed -- in fact, it is possible to make most of the alternatives an 
"optimal" solution by changing the weights. However, we proceed with the 
weights chosen, noting that we should verify the validity of the criteria 
weights if we need to make a similar decision in the future.

% You must have a Conclusions section
\section{Conclusions}

As shown in table~\ref{tbl:mcdm}, Amazon's S3 best meets the criteria outlined.

% Here is a table.  You MUST cite the table in the text before it appears
% in the document.
\begin{table}
  \caption{Multi-Criteria Decision Making Results.}
  \label{tbl:mcdm}
  \centering
  \begin{tabular}{|p{3.0cm}|p{0.75cm}|
                   p{0.75cm}|p{0.75cm}|p{0.75cm}|p{0.75cm}|p{0.75cm}|
                   p{0.75cm}|p{0.75cm}|p{0.75cm}|p{0.75cm}|p{0.75cm}|}
    \hline
    \textbf{Criteria} &
    \textbf{Wt.} &
    \multicolumn{2}{|p{1.50cm}|}{\textbf{S3}} &
    \multicolumn{2}{|p{1.50cm}|}{\textbf{Blobstore}} &
    \multicolumn{2}{|p{1.50cm}|}{\textbf{Storage API}} &
    \multicolumn{2}{|p{1.50cm}|}{\textbf{Imgur}} &
    \multicolumn{2}{|p{1.50cm}|}{\textbf{Dropbox}} \\
    \hline
    \hline
    Relationship &
       1 &  4 &  4 &  5 &  5 &  4 &  4 &  1 &  1 &  4 &  4 \\
    \hline
    Familiarity &
       3 &  4 & 12 &  5 &  15 &  2 &  6 &  1 &  3 &  2 &  6 \\
    \hline
    User Impact &
       3 &  4 & 12 &  1 &  3 &  1 &  3 &  5 & 15 &  4 & 12 \\
    \hline
    Tools &
       2 &  4 &  8 &  4 &  8 &  5 & 10 &  1 &  2 &  2 &  4 \\
    \hline
    APIs &
       2 &  5 & 10 &  4 &  8 &  4 &  8 &  2 &  4 &  3 &  6 \\
    \hline
    ACLs &
       2 &  4 &  8 &  1 &  2 &  5 & 10 &  1 &  2 &  3 &  6 \\
    \hline
    \hline
    \textbf{Total} &
      &
      \multicolumn{2}{|r|}{\textbf{54}} &
      \multicolumn{2}{|r|}{\textbf{41}} &
      \multicolumn{2}{|r|}{\textbf{41}} &
      \multicolumn{2}{|r|}{\textbf{27}} &
      \multicolumn{2}{|r|}{\textbf{38}} \\
    \hline
  \end{tabular}
\end{table}

% You must have a Recommendations section
\section{Recommendations}
We recommend that Khan Academy create an S3 bucket to store images for static 
questions using their existing Amazon Web Services account.

In particular, we note the ability to create cacheable URLs for objects stored 
on S3 by setting the "public-read" ACL on those objects.

%%%%%%%%%%%%%%%%%%%%%%%%%%%%%%%%%%%%%%%%%%%%%%%%%%%%%%%%%%%%%%%%%%%%%
%% BACK MATTER
%%%%%%%%%%%%%%%%%%%%%%%%%%%%%%%%%%%%%%%%%%%%%%%%%%%%%%%%%%%%%%%%%%%%%
%% \backmatter will make the \section commands ignore their numbering,
\backmatter

% Here, we insert a References section, which will be formatted properly.
% The list of works you have referenced should be specified in the preamble.
% In this template, the file is uw-wkrpt-bib.bib.
%
% Note, you will need to process the document in a certain order.  First,
% run LaTeX.  The % first pass will allow LaTeX to build a list of 
% references, it may % emit warning messages such as:
%   LaTeX Warning: Reference `app:gnugpl' on page 4 undefined on input line 277.
%   LaTeX Warning: There were undefined references.
% This is normal.  Now you run BiBTeX in order to generate the proper
% layout for the references.  After this, you run LaTeX once more.
\printbibliography[heading=bibintoc]

%%%%%%%%%%%%%%%%%%%%%%%%%%%%%%%%%%%%%%%%%%%%%%%%%%%%%%%%%%%%%%%%%%%%%
%% APPENDICES
%%%%%%%%%%%%%%%%%%%%%%%%%%%%%%%%%%%%%%%%%%%%%%%%%%%%%%%%%%%%%%%%%%%%%
%% \appendix will reset \section numbers and turn them into letters.
%%
%% Don't forget to refer to all your appendices in the main report.
\appendix
\begin{appendices}

\end{appendices}
\end{document}
